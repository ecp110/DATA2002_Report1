\documentclass[letterpaper,8pt,twocolumn,twoside,]{pinp}

%% Some pieces required from the pandoc template
\providecommand{\tightlist}{%
  \setlength{\itemsep}{0pt}\setlength{\parskip}{0pt}}

% Use the lineno option to display guide line numbers if required.
% Note that the use of elements such as single-column equations
% may affect the guide line number alignment.

\usepackage[T1]{fontenc}
\usepackage[utf8]{inputenc}

% pinp change: the geometry package layout settings need to be set here, not in pinp.cls
\geometry{layoutsize={0.95588\paperwidth,0.98864\paperheight},%
  layouthoffset=0.02206\paperwidth, layoutvoffset=0.00568\paperheight}

\definecolor{pinpblue}{HTML}{185FAF}  % imagecolorpicker on blue for new R logo
\definecolor{pnasbluetext}{RGB}{101,0,0} %


\usepackage{booktabs}
\usepackage{longtable}
\usepackage{array}
\usepackage{multirow}
\usepackage{wrapfig}
\usepackage{float}
\usepackage{colortbl}
\usepackage{pdflscape}
\usepackage{tabu}
\usepackage{threeparttable}
\usepackage{threeparttablex}
\usepackage[normalem]{ulem}
\usepackage{makecell}
\usepackage{xcolor}

\title{Executive Summary - DATA2002 Assignment 2}

\author[]{500520320, 480025267, 500505430, 500586901, 500555424}

  \affil[]{University of Sydney}

\setcounter{secnumdepth}{3}

% Please give the surname of the lead author for the running footer
\leadauthor{500520320, 480025267, 500505430, 500586901, 500555424}

% Keywords are not mandatory, but authors are strongly encouraged to provide them. If provided, please include two to five keywords, separated by the pipe symbol, e.g:
 

\begin{abstract}
The following paper analyses the quality of Vinho Verde red wine from
Northern Portugal, measured from a scale of one to ten. Using multiple
regression, this paper aims to model the quality of various red wines on
other factors such as the alcohol concentration or total sulfur dioxide
concentration. Various potential predictor variables were removed from
the model for not meeting key assumption criteria such as co-linearity
with red wine quality, normality, or residual homoscedasticity.
Transformations were attempted on variables that didn't meet the
co-linearity assumption, with the only successful transformation being
the log of total sulfur dioxide. Multiple regression was then computed
using both a step-back and step-forward AIC-based model. These both
netted the same model; a regression of red wine quality against alcohol
concentration, volatile acidity, the log of total sulfur dioxide and
density. In-sample performance of this model showcased an \(R^2\) value
of 0.32, indicating that the model may not be particularly strong, and
an RMSE of 0.669. Performing 10-fold cross validation suggested that our
selected model wasn't subject to obvious overfitting issues. The model
generated was as follows: \[
\mathbf{Quality = -22.22 + 0.32AC -0.67VA - 0.06log(TSD) + 23.31D}
\]
\end{abstract}

\dates{This version was compiled on \today} 


% initially we use doi so keep for backwards compatibility
% new name is doi_footer
\doifooter{\url{https://github.sydney.edu.au/epro3000/LAB-03-CC_early_8}}

\pinpfootercontents{LAB-03-CC\_early\_8}

\begin{document}

% Optional adjustment to line up main text (after abstract) of first page with line numbers, when using both lineno and twocolumn options.
% You should only change this length when you've finalised the article contents.
\verticaladjustment{-2pt}

\maketitle
\thispagestyle{firststyle}
\ifthenelse{\boolean{shortarticle}}{\ifthenelse{\boolean{singlecolumn}}{\abscontentformatted}{\abscontent}}{}

% If your first paragraph (i.e. with the \dropcap) contains a list environment (quote, quotation, theorem, definition, enumerate, itemize...), the line after the list may have some extra indentation. If this is the case, add \parshape=0 to the end of the list environment.

\acknow{We gratefully acknowledge all the help from the
\href{https://cran.r-project.org/package=rticles}{rticles} package
\citep{CRAN:rticles} which not only introduced us to the powerful
\href{http://www.pnas.org/site/authors/latex.xhtml}{PNAS LaTeX} style
class, but also provided useful code templates to study in the other
mode as the fine macros. The \href{http://pandoc.org}{pandoc} document
converter \citep{pandoc} is the much-admired driving force behind the
document manipulation. Additionally, we acknowledge the fantastic
support and teaching provided by G. Tarr and the DATA2002 team
throughout the semester.}

\hypertarget{introduction}{%
\section{Introduction}\label{introduction}}

\hypertarget{backgroud}{%
\subsection{\texorpdfstring{\texttt{Backgroud}}{Backgroud}}\label{backgroud}}

This dataset highlights different factors which contribute to an overall
quality rating of the Portuguese red wine ``Vinho Verde''. Red wine
quality may be of interest to producers of wine who are trying to better
understand the factors that contribute to overall quality, as well as
consumers who are interested in specific characteristics of red wines
from speciality regions such as Portugal.

\hypertarget{dataset-overview}{%
\subsection{\texorpdfstring{\texttt{Dataset\ Overview}}{Dataset Overview}}\label{dataset-overview}}

The dataset had 1599 rows corresponding to different wine observations,
and was collected in 2009 from ``Vinho Verde'' wine in the north of
Portugal. The input variables presented to us were collected from
chemical tests, and consisted of fixed acidity, volatile acidity, citric
acid, residual sugar, chlorides, free sulfur dioxide, total sulfur
dioxide, density, pH, sulphates, and alcohol. The output variable is the
wine quality, graded on a scale of 0 to 10. This was determined by the
median score of at least three evaluations made by wine experts.

The first of the 12 variables is the fixed acidity. The fixed acidity is
a measure of the tartaric, malic, citric and succinic, found within
grapes, which are measured by a steam distillation of a sample of the
wine. The volatile acidity (being the second variable) however measures
the acetic acid levels in the wine which impacts on the flavour of the
wine negatively if the acetic acid is too high. The third variable is
citric acid which is used in wine in order to assist in increasing
flavour by in turn increasing acidity. Moving away from acidity now, the
data set analyses the residual sugars in ``Vinho Verde''. The residual
sugar in wines is not the artificially added sugars, but the raw sugar
from grapes found post fermentation, with higher levels of residual
sugar unsurprisingly resulting in sweeter wine. Chlorides are used in
this data set as a factor contributing towards an overall rating for the
wine due to its impact on the saltiness of the final product. The
density of wine is deemed to be a major contributor to the overall
quality of wine, with higher density resulting in higher quality. pH is
simply used in wine making in order to ``determine ripeness in relation
to acidity''. Lower pH is far more desirable as it reduces
susceptibility to bacteria growth, while also increasing the quality of
taste by making the final product ``tart and crisp''. Sulphate levels
are the next criteria the researchers used to determine overall wine
quality rankings, with the sulphate being used in wine in order to
protect the final product from bacteria. Evidently, alcohol
concentration is a major contributing factor towards the final quality
of ``Vinho Verde'', with it having a large impact on flavour and
desirability for the intended market. Finally, the final quality ranking
summarises all of the previous factors into a final value.

\hypertarget{analysis}{%
\section{Analysis}\label{analysis}}

\hypertarget{assumption-validation}{%
\subsection{\texorpdfstring{\texttt{Assumption\ Validation}}{Assumption Validation}}\label{assumption-validation}}

Multiple regression relies upon four key assumptions: 1. Linearity. 2.
Independence. 3. Normality. 4. Homoscedasticity.

For each of these assumptions, the individual relationships between the
input variables and quality of wine must meet the assumptions
themselves, as well as the final model, and thus were checked
individually for each input variable against the quality of wine. For
independence, collection methodologies were not explicitly outlined in
the dataset, however we can assume that each wine was independent of
each other for the purposes of this report.

Linearity refers to how well the input variables can be linearly mapped
to the output variable. Violations included free sulphur dioxide, fixed
acidity and pH, the latter of which is shown as an example here. {[}{]}
is also shown as an example of meeting the linearity assumption.
Additionally, for any variable which violated the linearity assumptions,
data transformations were attempted to improve linearity. The only
successful case of this was the log transformation of Total Sulphur
Dioxide, as it allowed for the multiple orders of magnitude of total
sulphur dioxide to be better expressed linearly against wine quality.
(FIgure xxx)

Normality refers to the fact that residuals should be normally
distributed. Violations included residual sugar which can be seen below,
as compared to a normal example of {[}{]}.

Finally, homoscedasticity refers to the even distribution of the
residuals around the mean. Any fanning of the residuals suggests
heteroscedasticity, violating this assumption. Violations include
sulphates and chlorides as they displayed distinct residual distribution
patterns. Variables which violated assumptions were excluded from the
dataset used to construct the model. As such, we were left with alcohol,
volatile acidity, the log of total sulphur dioxide, density and citric
acid as the remaining input variables which met all assumptions.

\hypertarget{model-selection}{%
\subsection{\texorpdfstring{\texttt{Model\ Selection}}{Model Selection}}\label{model-selection}}

Both step-forward and step-back AIC variable selection methods for
building a multiple regression model of the quality of red wine came to
the same conclusion as can be shown in table 1 and table 2 below. Of the
predictor variables that were used to build the model, i.e.~those that
met the fundamental multiple regression assumptions outlined previously,
only the predictors in the two tables were significant. Therefore, the
combination of predictors we are using to predict the quality of red
wine are the alcohol concentration, volatile acidity, total sulphur
dioxide, and the density.

Hence, our model is as follows:

\[
\mathbf{Quality = -22.22 + 0.32AC -0.67VA - 0.06log(TSD) + 23.31D}
\] This model had an AIC of 3259.415, an \(R^2\) of 0.32 and an RMSE of
0.669.

\begin{table}[!h]

\caption{\label{tab:unnamed-chunk-2}Step-Forward Method}
\centering
\begin{tabular}[t]{lrrrr}
\toprule
  & Estimate & Std. Error & t value & Pr(>|t|)\\
\midrule
Intercept & -21.22 & 10.31 & -2.06 & 0.04\\
Alcohol Concentration & 0.32 & 0.02 & 16.99 & 0.00\\
Volatile Acidity & -0.67 & 0.05 & -13.64 & 0.00\\
Total Sulfur Dioxide & -0.06 & 0.02 & -2.29 & 0.02\\
Density & 23.31 & 10.25 & 2.27 & 0.02\\
\bottomrule
\end{tabular}
\end{table}

\begin{table}[!h]

\caption{\label{tab:unnamed-chunk-2}Step-Back Method}
\centering
\begin{tabular}[t]{lrrrr}
\toprule
  & Estimate & Std. Error & t value & Pr(>|t|)\\
\midrule
Intercept & -21.22 & 10.31 & -2.06 & 0.04\\
Alcohol Concentration & 0.32 & 0.02 & 16.99 & 0.00\\
Volatile Acidity & -0.67 & 0.05 & -13.64 & 0.00\\
Total Sulfur Dioxide & -0.06 & 0.02 & -2.29 & 0.02\\
Density & 23.31 & 10.25 & 2.27 & 0.02\\
\bottomrule
\end{tabular}
\end{table}

\hypertarget{results-discussion}{%
\section{Results \& Discussion}\label{results-discussion}}

\hypertarget{in-sample-performance}{%
\subsection{\texorpdfstring{\texttt{In\ Sample\ Performance}}{In Sample Performance}}\label{in-sample-performance}}

As stated, the selected model had an \(R^2\) of 0.32 and an RMSE of
0.669. This means that approximately 32\% of the total variance in red
wine quality can be explained by our regression model. This isn't
particularly high, indicating this linear model may not be performing
strongly. This could be due to the somewhat poor co-linearity observed
with most predictor variables versus wine quality in this dataset.
However, to ensure that our model isn't subject to overfitting, we want
to test it using out of sample performance.

\hypertarget{out-of-sample-performance}{%
\subsection{\texorpdfstring{\texttt{Out\ of\ Sample\ Performance}}{Out of Sample Performance}}\label{out-of-sample-performance}}

Using the caret package, we performed a 10-fold cross validation. The
purpose of this was to test how well our model with predictor variables
of alcohol concentration, volatile acidity, the log of total sulphur
dioxide and density performed out of sample. This method allows for
error to only be judged based on performance on the testing subset of
the original sample, such that our judgement on the performance of the
model isn't impacted by the data that was used to produce the model in
the first place.

Using 10-fold cross validation, we divided our sample randomly into 10
folds, of which one was allocated to be the testing set, while the rest
were training folds used to build the model using the significant
predictor variables. This process was repeated 10 times, where a
different fold would be the testing set each time. The \(R^2\), root
mean squared error and mean average error of each prediction, compared
to the real value in the testing set was then computed.

The following graph shows the distribution of \(R^2\), MAE and RMSE for
each of the 10 total cross validations performed. We have compared the
performance of three separate models. The `full' model represents all of
the potential predictors that were initially present in the red wine
quality data frame -- even those that didn't meet assumptions such as
linearity. The `selected' model represents the predictor variables we
chose that were outlined previously. Finally, the simple model was the
regression of red wine quality using only alcohol concentration which
was the most significant predictor.

The full model had a median \(R^2\) of 0.35, which was higher than the
median \(R^2\) of both the selected and simple models. However, as the
full model had predictors that violated assumptions, it is an
inappropriate multiple regression model so we must take this result with
a grain of salt. The higher \(R^2\) is likely to do with the fact that
the larger number of predictors used would result in overfitting of the
data, and artificially increasing the \(R^2\) even though the
relationship between the wine quality and some of the predictors may not
have been linear in the first place. In comparison, the median \(R^2\)
of our model was 0.32, which was higher than that of the simple model
which was 0.23. Interestingly, as the \(R^2\) of our selected model on
the training data was very similar to the \(R^2\) during in-sample
performance, it appears that our selected model didn't suffer from
overfitting.

Again, discounting the full model, the selected model had the lowest
median MAE 0.527 and RMSE (0.671) of the models being compared,
indicating that it performed better on the testing set of red wine
quality compared to the simple model. (figure name)

\hypertarget{assumption-re-validation}{%
\subsection{\texorpdfstring{\texttt{Assumption\ Re-Validation}}{Assumption Re-Validation}}\label{assumption-re-validation}}

Additionally, our multiple regression model was also checked against the
key assumptions outlined previously -- normality and homoscedasticity.
The following figures show that there is no obvious fanning pattern in
the residuals, indicating homoscedasticity, and that the residuals of
the full model are also normally distributed.

\hypertarget{limitations}{%
\subsection{\texorpdfstring{\texttt{Limitations}}{Limitations}}\label{limitations}}

This model has a few limitations. The linearity of most predictor
variables against red wine quality, even after attempting
transformations, wasn't very clear and so we had to be generous when
determining what met this assumption for the purpose of analysis. An
\(R^2\) of 0.32 for the selected model is fairly low, which indicates
that the in-sample performance of the model isn't particularly strong,
even though it appears the model didn't run into overfitting issues when
performing 10-fold cross-validation

There is also not much information available on the data collection
process, so we are unsure about whether the different observations are
independent or not. For example, red wines from neighbouring regions may
not be independent and the quality of wine could be influenced as such.
This is an assumption we have had to make for the purpose of the
analysis.

\hypertarget{conclusion}{%
\section{Conclusion}\label{conclusion}}

Overall, we were able to model the quality of red wine by linear
regression, using alcohol concentration, volatile acidity, log of total
sulfur dioxide, and density. We determined from this that there was a
positive relationship between wine quality and alcohol quantity as well
as density, whereas a negative relationship was determined between wine
quality and volatile acidity as well as total sulphur dioxide.

This could be of interest to both consumers of wine who are interested
in characteristics of wine that may determine quality, as well as
producers who are trying to make wines of certain qualities at certain
price ranges.

\hypertarget{references}{%
\section{References}\label{references}}

%\showmatmethods
\showacknow




\begin{thebibliography}{3}
\newcommand{\enquote}[1]{``#1''}
\providecommand{\natexlab}[1]{#1}
\providecommand{\url}[1]{\texttt{#1}}
\providecommand{\urlprefix}{URL }
\expandafter\ifx\csname urlstyle\endcsname\relax
  \providecommand{\doi}[1]{doi:\discretionary{}{}{}#1}\else
  \providecommand{\doi}{doi:\discretionary{}{}{}\begingroup
  \urlstyle{rm}\Url}\fi
\providecommand{\eprint}[2][]{\url{#2}}

\bibitem[{Allaire \emph{et~al}(2017)Allaire, {R Foundation}, Wickham, {Journal
  of Statistical Software}, Xie, Vaidyanathan, {Association for Computing
  Machinery}, Boettiger, {Elsevier}, Broman, Mueller, Quast, Pruim, Marwick,
  Wickham, Keyes, and Yu}]{CRAN:rticles}
Allaire J, {R Foundation}, Wickham H, {Journal of Statistical Software}, Xie Y,
  Vaidyanathan R, {Association for Computing Machinery}, Boettiger C,
  {Elsevier}, Broman K, Mueller K, Quast B, Pruim R, Marwick B, Wickham C,
  Keyes O, Yu M (2017).
\newblock \emph{rticles: Article Formats for R Markdown}.
\newblock R package version 0.4.1,
  \urlprefix\url{https://CRAN.R-project.org/package=rticles}.

\bibitem[{MacFarlane J (2017)}]{pandoc}
MacFarlane J (2017).
\newblock \emph{Pandoc: A Universal Document Converter}.
\newblock Version 1.19.2.1, \urlprefix\url{http://pandoc.org}.

\bibitem[{Xie Y (2017)}]{}
Xie Y (2017).
\newblock \emph{knitr: A General-Purpose Package for Dynamic Report Generation
  in R}.
\newblock R package version 1.17, \urlprefix\url{https://yihui.name/knitr/}.

\bibitem[{Wickham et al., (2019)}]{}
Wickham et al., (2019).
\newblock \emph{Welcome to the tidyverse. Journal of Open
Source Software, 4(43), 1686}.
\newblock \urlprefix\url{https://doi.org/10.21105/joss.01686}.

\bibitem[{McLean MW(2017)}]{}
McLean MW (2017).
\newblock \emph{RefManageR: Import and Manage BibTeX and BibLaTeX
References in R. The Journal of Open Source Software}.
\newblock \urlprefix\url{https://doi.org/10.21105/joss.00338}.

\bibitem[{McLean MW (2014)}]{}
McLean MW (2014).
\newblock \emph{Straightforward Bibliography Management in R
Using the RefManager Package. arXiv: 1403.2036 [cs.DL]{}}.
\newblock \urlprefix\url{https://arxiv.org/abs/1403.2036}.

\bibitem[{Schloerke B, Cook D, Larmarange J, Briatte F,
Marbach M, Thoen E, Elberg T and Crowley J (2021)}]{}
Schloerke B, Cook D, Larmarange J, Briatte F,
Marbach M, Thoen E, Elberg T and Crowley J (2021).
\newblock \emph{GGally: Extension to ggplot2}.
\newblock R package version 2.1.2. \urlprefix\url{https://CRAN.R-project.org/package=GGally}.

\bibitem[{Fox J and Weisberg S (2019)}]{}
Fox J and Weisberg S (2019).
\newblock \emph{R Companion to Applied
Regression, Third Edition. Thousand Oaks CA: Sage}.
\newblock \urlprefix\url{https://socialsciences.mcmaster.ca/jfox/Books/Companion/}.

\bibitem[{Tang Y, Horijoshi M and Li W (2016)}]{}
Tang Y, Horijoshi M and Li W (2016)
\newblock \emph{ggfortify: Unified
Interface to Visualize Statistical Result of Popular R Packages}.
\newblock The R Journal 8.2 (2016): 478-489.

\bibitem[{Lüdecke D (2021)}]{}
Lüdecke D (2021)
\newblock \emph{sjPlot: Data Visualization for Statistics in
Social Science}.
\newblock R package version 2.8.9 \urlprefix\url{https://CRAN.R-project.org/package=sjPlot}.

\bibitem[{Anderson D, Heiss A, Sumners J (2021)}]{}
Anderson D, Heiss A, Sumners J (2021).
\newblock \emph{equatiomatic: Transform Models into LaTeX Equations}.
\newblock R package version 0.3.0 \urlprefix\url{https://CRAN.R-project.org/package=equatiomatic}.

\bibitem[{Faraway J (2016)}]{}
Faraway J (2016).
\newblock \emph{faraway: Functions and Datasets for Books}.
\newblock R package version 1.0.7 \urlprefix\url{https://CRAN.R-project.org/package=faraway}.

\bibitem[{Broman K(2015)}]{}
Broman K (2015).
\newblock \emph{R/qtlcharts: interactive graphics for
quantitative trait locus mapping}.
\newblock \urlprefix\url{doi:10.1534/genetics.114.172742}.

\bibitem[{Kuhn M (2021)}]{}
Kuhn M (2021).
\newblock \emph{caret: Classification and Regression Training}.
\newblock R package version 6.0-90. \urlprefix\url{https://CRAN.R-project.org/package=caret}.

\bibitem[{Zhu H (2021)}]{}
Zhu H (2021).
\newblock \emph{ableExtra: Construct Complex Table with 'kable' and Pipe Syntax}.
\newblock R package version 1.3.4. \urlprefix\url{https://CRAN.R-project.org/package=kableExtra}..

\bibitem[{Firke S (2021)}]{}
Firke S (2021).
\newblock \emph{Janitor: Simple Tools for Examining and Cleaning Dirty Data}.
\newblock R package version 2.1.0. \urlprefix\url{https://CRAN.R-project.org/package=janitor}.

\bibitem[{Wickham H, François R, Henry L and Müller K (2021)}]{}
Wickham H, François R, Henry L and Müller K (2021).
\newblock \emph{dplyr: A Grammar of Data Manipulation}.
\newblock R package version 1.0.7. \urlprefix\url{https://CRAN.R-project.org/package=dplyr}.

\bibitem[{Lumley T(2020)}]{}
Lumley T(2020).
\newblock \emph{leaps: Regression Subset Selection}.
\newblock  R package version 3.1.0. \urlprefix\url{https://CRAN.R-project.org/package=leaps}.

\bibitem[{Tarr G (2021)}]{}
Tarr G (2021).
\newblock \emph{Week three Lecture L08: Testing for independence – who was more likely to die on the Titanic?}.
\newblock [Powerpoint Slides]{}. DATA2002, University of Sydney, Sydney, Australia. 

\bibitem[{Tarr G (2021)}]{}
Tarr G (2021).
\newblock \emph{Week eleven Lecture L26: Simple linear regression}.
\newblock [Powerpoint Slides]{}. DATA2002, University of Sydney, Sydney, Australia. 

\bibitem[{Tarr G (2021)}]{}
Tarr G (2021).
\newblock \emph{Week eleven Lecture L27: Multiple regression and model selection}.
\newblock [Powerpoint Slides]{}. DATA2002, University of Sydney, Sydney, Australia. 

\bibitem[{Tarr G (2021)}]{}
Tarr G (2021).
\newblock \emph{Week eleven Lecture L28: Prediction internals and performance assessment}.
\newblock [Powerpoint Slides]{}. DATA2002, University of Sydney, Sydney, Australia. 

\bibitem[{Tarr G (2021)}]{}
Tarr G (2021).
\newblock \emph{Week ten Live Lecture}.
\newblock [Video file]{}. DATA2002, University of Sydney, Sydney, Australia. 

\bibitem[{Tarr G (2021)}]{}
Tarr G (2021).
\newblock \emph{Week eleven Live Lecture}.
\newblock [Video file]{}. DATA2002, University of Sydney, Sydney, Australia. 

\end{thebibliography}

\end{document}
