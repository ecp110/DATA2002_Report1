\documentclass[letterpaper,9pt,twocolumn,twoside,]{pinp}

%% Some pieces required from the pandoc template
\providecommand{\tightlist}{%
  \setlength{\itemsep}{0pt}\setlength{\parskip}{0pt}}

% Use the lineno option to display guide line numbers if required.
% Note that the use of elements such as single-column equations
% may affect the guide line number alignment.

\usepackage[T1]{fontenc}
\usepackage[utf8]{inputenc}

% pinp change: the geometry package layout settings need to be set here, not in pinp.cls
\geometry{layoutsize={0.95588\paperwidth,0.98864\paperheight},%
  layouthoffset=0.02206\paperwidth, layoutvoffset=0.00568\paperheight}

\definecolor{pinpblue}{HTML}{185FAF}  % imagecolorpicker on blue for new R logo
\definecolor{pnasbluetext}{RGB}{101,0,0} %



\title{Executive Summary}

\author[a]{Lab-03-CC}
\author[a]{500520320, 480025267, 00505430, 500586901, 500555424}

  \affil[a]{University of Sydney}

\setcounter{secnumdepth}{3}

% Please give the surname of the lead author for the running footer
\leadauthor{Eddelbuettel and Balamuta}

% Keywords are not mandatory, but authors are strongly encouraged to provide them. If provided, please include two to five keywords, separated by the pipe symbol, e.g:
 

\begin{abstract}
This short vignette details several of the available options for the
\texttt{pinp} pdf vignette template.
\end{abstract}

\dates{This version was compiled on \today} 


% initially we use doi so keep for backwards compatibility
% new name is doi_footer
\doifooter{\url{https://cran.r-project.org/package=pinp}}

\pinpfootercontents{pinp Vignette}

\begin{document}

% Optional adjustment to line up main text (after abstract) of first page with line numbers, when using both lineno and twocolumn options.
% You should only change this length when you've finalised the article contents.
\verticaladjustment{-2pt}

\maketitle
\thispagestyle{firststyle}
\ifthenelse{\boolean{shortarticle}}{\ifthenelse{\boolean{singlecolumn}}{\abscontentformatted}{\abscontent}}{}

% If your first paragraph (i.e. with the \dropcap) contains a list environment (quote, quotation, theorem, definition, enumerate, itemize...), the line after the list may have some extra indentation. If this is the case, add \parshape=0 to the end of the list environment.

\acknow{We gratefully acknowledge all the help from the
\href{https://cran.r-project.org/package=rticles}{rticles} package
\citep{CRAN:rticles} which not only introduced us to the powerful
\href{http://www.pnas.org/site/authors/latex.xhtml}{PNAS LaTeX} style
class, but also provided useful code templates to study in the other
mode as the fine macros. The \href{http://pandoc.org}{pandoc} document
converter \citep{pandoc} is the much-admired driving force behind the
document manipulation.}

\hypertarget{abstract}{%
\section{Abstract}\label{abstract}}

An \emph{optional} selection (via value \texttt{true}) of one-sided
rather than two-sided output. This should probably alter the footnote
but does not currently do so.

An \emph{optional} selection (via value \texttt{true}) of one-sided
rather than two-sided output. This should probably alter the footnote
but does not currently do so.

\hypertarget{introduction}{%
\section{Introduction}\label{introduction}}

\hypertarget{backgroud}{%
\subsection{\texorpdfstring{\texttt{Backgroud}}{Backgroud}}\label{backgroud}}

An \emph{optional} selection (via value \texttt{true}) of one-sided
rather than two-sided output. This should probably alter the footnote
but does not currently do so.

\hypertarget{data-set-description}{%
\section{Data set description}\label{data-set-description}}

insert description

An \emph{optional} selection (via value \texttt{true}) of one-sided
rather than two-sided output. This should probably alter the footnote
but does not currently do so.An \emph{optional} selection (via value
\texttt{true}) of one-sided rather than two-sided output. This should
probably alter the footnote but does not currently do so.

\hypertarget{analysis}{%
\section{Analysis}\label{analysis}}

\hypertarget{assumption-checkings}{%
\subsection{\texorpdfstring{\texttt{Assumption\ checkings}}{Assumption checkings}}\label{assumption-checkings}}

An \emph{optional} selection (via value \texttt{true}) of one-sided
rather than two-sided output. This should probably alter the footnote
but does not currently do so.

\hypertarget{model-selection}{%
\subsection{\texorpdfstring{\texttt{Model\ selection}}{Model selection}}\label{model-selection}}

An \emph{optional} selection (via value \texttt{true}) of one-sided
rather than two-sided output. This should probably alter the footnote
but does not currently do so.

\hypertarget{results}{%
\section{Results}\label{results}}

\hypertarget{inferences}{%
\subsection{\texorpdfstring{\texttt{Inferences}}{Inferences}}\label{inferences}}

Equation explaination

An \emph{optional} selection (via value \texttt{true}) of one-sided
rather than two-sided output. This should probably alter the footnote
but does not currently do so.

\hypertarget{performance}{%
\subsection{\texorpdfstring{\texttt{Performance}}{Performance}}\label{performance}}

An \emph{optional} selection (via value \texttt{true}) of one-sided
rather than two-sided output. This should probably alter the footnote
but does not currently do so.An \emph{optional} selection (via value
\texttt{true}) of one-sided rather than two-sided output. This should
probably alter the footnote but does not currently do so.

\hypertarget{dicussion}{%
\section{Dicussion}\label{dicussion}}

\hypertarget{limitation}{%
\subsection{\texorpdfstring{\texttt{Limitation}}{Limitation}}\label{limitation}}

An \emph{optional} selection (via value \texttt{true}) of one-sided
rather than two-sided output. This should probably alter the footnote
but does not currently do so.

\hypertarget{conclusion}{%
\section{Conclusion}\label{conclusion}}

An \emph{optional} selection (via value \texttt{true}) of one-sided
rather than two-sided output. This should probably alter the footnote
but does not currently do so.

\hypertarget{reference}{%
\section{Reference}\label{reference}}

%\showmatmethods
\showacknow




\begin{thebibliography}{3}
\newcommand{\enquote}[1]{``#1''}
\providecommand{\natexlab}[1]{#1}
\providecommand{\url}[1]{\texttt{#1}}
\providecommand{\urlprefix}{URL }
\expandafter\ifx\csname urlstyle\endcsname\relax
  \providecommand{\doi}[1]{doi:\discretionary{}{}{}#1}\else
  \providecommand{\doi}{doi:\discretionary{}{}{}\begingroup
  \urlstyle{rm}\Url}\fi
\providecommand{\eprint}[2][]{\url{#2}}

\bibitem[{Allaire \emph{et~al.}(2017)Allaire, {R Foundation}, Wickham, {Journal
  of Statistical Software}, Xie, Vaidyanathan, {Association for Computing
  Machinery}, Boettiger, {Elsevier}, Broman, Mueller, Quast, Pruim, Marwick,
  Wickham, Keyes, and Yu}]{CRAN:rticles}
Allaire J, {R Foundation}, Wickham H, {Journal of Statistical Software}, Xie Y,
  Vaidyanathan R, {Association for Computing Machinery}, Boettiger C,
  {Elsevier}, Broman K, Mueller K, Quast B, Pruim R, Marwick B, Wickham C,
  Keyes O, Yu M (2017).
\newblock \emph{rticles: Article Formats for R Markdown}.
\newblock R package version 0.4.1,
  \urlprefix\url{https://CRAN.R-project.org/package=rticles}.

\bibitem[{MacFarlane(2017)}]{pandoc}
MacFarlane J (2017).
\newblock \emph{Pandoc: A Universal Document Converter}.
\newblock Version 1.19.2.1, \urlprefix\url{http://pandoc.org}.

\bibitem[{Xie(2017)}]{CRAN:knitr}
Xie Y (2017).
\newblock \emph{knitr: A General-Purpose Package for Dynamic Report Generation
  in R}.
\newblock R package version 1.17, \urlprefix\url{https://yihui.name/knitr/}.

\end{thebibliography}

\end{document}
